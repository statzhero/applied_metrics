\documentclass[aspectratio=169]{beamer}
\usetheme{metropolis}

%\usepackage{beamerthemesplit}
%\beamertemplatenavigationsymbolsempty
\usepackage{amsmath}
\usepackage{amsthm}
\usepackage{amssymb}
\usepackage{latexsym}
\usepackage{graphicx}
\usepackage{fancybox}
\usepackage{dsfont}
\usepackage{multirow} 
\usepackage{multicol}
\usepackage{booktabs} 
\usepackage{dcolumn}
\usepackage{soul}
%\usepackage[cache=false]{minted}
\usepackage{MnSymbol}
\usepackage{stmaryrd}


\DeclareMathOperator*{\argmax}{arg\,max}
\DeclareMathOperator*{\argmin}{arg\,min}

\newcommand{\X}{\mathtt{X}}
\newcommand{\Y}{\mathtt{Y}}

%\newcommand{\R}{\mathbb{R}}
%\newcommand{\E}{\mathbb{E}}
%\newcommand{\V}{\mathbb{V}}
\newcommand{\p}{\mathbb{P}}
\newcommand*\df{\mathop{}\!\mathrm{d}}
\newcommand{\del}{\partial}


% imports
\usepackage{xargs}
\usepackage{xpatch}
\usepackage{etoolbox}
\usepackage{pdflscape}
\usepackage{booktabs}
\usepackage{threeparttable}
\usepackage[skip=0.2\baselineskip]{caption}

% command for inputting raw latex
\makeatletter
\newcommand\primitiveinput[1]{\@@input #1 }
\makeatother

% common table command
\newcommandx{\tablecontent}[4]{
    \begin{threeparttable}[!ht]
        \centering
        \caption{#3}
        \vspace{-1em}
        \footnotesize
        \begin{tabular}{#1}
            \primitiveinput{../tables/#2.tex}
        \end{tabular}
        \vspace{-0.2em}
        \begin{tablenotes}[flushleft]
            #4
        \end{tablenotes}
    \end{threeparttable}
}




% \usepackage{slashbox}
\title{Lecture 3: Generalized Method of Moments}
\author{Chris Conlon }
\institute{NYU Stern }


\newcommand{\norm}[1]{\left\lVert#1\right\rVert}
\newcommand{\R}{\mathbb{R}}
\newcommand{\E}{\mathbb{E}}
\newcommand{\V}{\mathbb{V}}
\newcommand{\ol}{\overline}
%\newcommand{\ul}{\underline}
\newcommand{\pp}{{\prime \prime}}
\newcommand{\ppp}{{\prime \prime \prime}}
\newcommand{\policy}{\gamma}


\newcommand{\fp}{\frame[plain]}

\date{\today}

\begin{document}
\maketitle


 

\begin{frame}{Review Definitions}

\begin{itemize}
\item Sample average moment conditions: $g_N(\theta) \in \mathbb{R}^q$, where $g_N(\theta)$ is a $q$-dimensional vector of moment conditions.
\begin{eqnarray*}
\mathbb{E}[g(w_i, \theta )] \approx \frac{1}{N} \sum_{i=1}^N g(w_i, \theta)  \equiv g_N(\theta)
\end{eqnarray*}
\item At the truth $\theta_0$: $\mathbb{E}[g(w_i, \theta_0 )] =\mathbf{0}_q$
\item Choose $\widehat{\theta}_{gmm}$ to minimize $$Q_N(\theta)=g_N(\theta)' \cdot W_N \cdot  g_N(\theta)$$
\item Have to choose a weighting matrix $W_n$.
\item Jacobian: $D(\theta) \equiv \mathbb{E}[\frac{\partial g(w_i,\theta)}{\partial \theta}]$, which is a $q \times k$ matrix.
\item  Evaluated at the optimum, $\frac{1}{\sqrt{N}} \sum_{i=1}^N g(w_i,\theta_0) \overset{d}{\to} N(0,S)$ where $S = E[g(w_i,\theta_0) g(w_i,\theta_0)']$ is a $q \times q$ matrix.
\end{itemize}
\end{frame}


\begin{frame}{GMM: Linear IV}
For the linear IV problem this becomes:
\begin{eqnarray*}
g_N(\theta)' W_N  g_N(\theta) &=& \frac{1}{N^2} \cdot (\mathbf{Z}' (\mathbf{Y} - \mathbf{X} \beta))' W_N (\mathbf{Z}' (\mathbf{Y} - \mathbf{X} \beta)) \\
&=& \frac{1}{N^2}\cdot [Y'Z W_N Z' Y - 2 \beta X' Z W_N Z' Y + \beta' X' Z W_N Z' X \beta]
\end{eqnarray*}
We can ignore the $\frac{1}{N^2}$ and take the first-order condition:
\begin{eqnarray*}
2 X'Z W_N Z' Y &=& 2 X'Z W_N Z' X \beta\\
\hat{\beta}_{GMM} &=& (X'Z W_N Z' X)^{-1} X' Z W_N Z'Y
\end{eqnarray*}
\alert{Hopefully this looks familiar}
\end{frame}

\begin{frame}{GMM: OLS}
\begin{itemize}
\item Suppose that we do not have any excluded instruments so that $Z=X$ (and thus $q=k$). 
\item Also suppose that $W_N = \mathbf{I}_q$ (the identity matrix). 
\item Then we can see that:
\begin{eqnarray*}
\hat{\beta}_{GMM} &=& (X'X \mathbf{I}_q X' X)^{-1} X' X \mathbf{I}_q X'Y\\
 &=& (X'X X' X)^{-1} X' X X'Y\\
 &=& (X'X)^{-1} (X' X)^{-1} (X' X) X'Y =  (X'X)^{-1} X'Y = \hat{\beta}_{OLS}
\end{eqnarray*}

\item In other words, OLS is a special case of the GMM estimator.
\item  Also, the identification condition $D=\frac{\partial g(w_i,\theta)}{\partial \theta} =\frac{1}{N} \sum_{i=1}^N z_i' x_i = \frac{1}{N} \sum_{i=1}^N x_i' x_i$ becomes that $rank(X'X) = k$ the well-known OLS rank condition.\\
\end{itemize}
\end{frame}

\begin{frame}{2SLS Estimator}
Suppose that we do have excluded instruments so that $dim(Z) = q > dim(X) = k$ and that $W_N = (Z'Z)^{-1}$. It immediately follows that:
\begin{eqnarray*}
\hat{\beta}_{GMM} &=& (X'Z (Z'Z)^{-1} Z' X)^{-1} X' Z (Z'Z)^{-1} Z'Y = \hat{\beta}_{2SLS}
\end{eqnarray*}
If $dim(Z) = q = dim(X) = k$ then $(X'Z)$ is square (and invertible). This expression further simplifies:
\begin{eqnarray*}
(X'Z (Z'Z)^{-1} Z' X)^{-1} = (Z'X)^{-1} (Z'Z) (X'Z)^{-1}\\
\rightarrow \hat{\beta}_{GMM} = (Z'X)^{-1} (Z'Z) (X'Z)^{-1}  X' Z (Z'Z)^{-1} Z'Y = (Z'X)^{-1} Z'Y = \hat{\beta}_{IV}
\end{eqnarray*}
\end{frame}

\begin{frame}{Efficient GMM}
An important question remains how one should choose the weighting matrix $W_N$. We've already seen two options: 
\begin{enumerate}
\item The identity matrix $\mathbf{I}_q$ equally penalizes violations of all $q$ moments
\item  the TSLS weighting matrix $(Z'Z)^{-1}$ which can be thought about as the inverse of the covariance of the instruments. 
\item The choice of weighting matrix only matters in the \alert{overidentified} case $q > k$. Why?
\end{enumerate}
We are interested in \alert{efficient GMM} which is the GMM estimator with the lowest variance.

\end{frame}

\begin{frame}{Efficient GMM}
 In order to find the $W_N$ which minimizes the variance of $\hat{\theta}_{GMM}$ we recall the asymptotic variance of the GMM estimator:
\begin{eqnarray*}
V_{\theta} =(D W D')^{-1} (D W S W' D') (D W D')^{-1}
\end{eqnarray*}
It turns out that the best choice of $W_N = S^{-1}$ (which sets $filling = bread$). This is easy to see, because $W_N$ is positive semi-definite.
\begin{eqnarray*}
(D S^{-1} D')^{-1} (D S^{-1} S S^{-1'} D') (D S^{-1} D')^{-1} = (D S^{-1} D')^{-1} (D S^{-1} D') (D S^{-1} D')^{-1} = (D S^{-1} D')^{-1}
\end{eqnarray*}
\end{frame}

\begin{frame}{Efficient GMM: Discussion}
This gives us some insight into what we are looking for from moment conditions. 

\begin{itemize}
\item We want $S$ to be small (we want the sampling variation/noise of our moments to be as small as possible). 
\item We also want $D$ (the Jacobian of the moments) to be large. 
\begin{itemize}
\item This means that small violations in moment conditions lead to large changes in the objective function. 
\item In practical terms, the problem is well identified when the objective function is steep around $\theta_0$.
\item When the problem becomes flat, it becomes hard to distinguish one $\theta$ in favor of another.
\end{itemize}
\item The problem is that $S = E[g(w_i,\theta_0) g(w_i,\theta_0)']$ is not something that we readily observe from our data. In fact, the asymptotic covariance evaluated at $\theta_0$ is \alert{infeasible}.
\end{itemize}
\end{frame}

\begin{frame}{Efficient GMM: Feasible Weight Matrix}
The best we can hope for is to use some sample analogue $W_N=\hat{S}^{-1}$ in its place. One way to compute that is the covariance of the moments estimated at some $\hat{\theta}$ for an initial guess of $W$:
\begin{eqnarray*}
\hat{W} = \hat{S}^{-1} = \left(\frac{1}{N} \sum_{i=1}^N (g(w_i,\hat{\theta}) - g_N(\hat{\theta}))  \, (g(w_i,\hat{\theta}) - g_N(\hat{\theta}))'\right)^{-1}
\end{eqnarray*}
Because $E[g(w_i,\theta_0)]=0$ at $\theta_0$ there is a tendency to use $\left(\frac{1}{N} \sum_{i=1}^N g(w_i,\hat{\theta}) \, g(w_i,\hat{\theta} )'\right)^{-1}$ (without de-meaning the moments). In theory this would work fine, but in practice \alert{it is nearly always a bad idea}.\\
\end{frame}

\begin{frame}{Efficient GMM: Review}

\noindent The overall procedure works as follows:
\begin{enumerate}
\item Pick some initial weighting matrix $W_0$: often $\mathbf{I}_q$ or $(Z'Z)^{-1}$.
\item Solve $\hat{\theta} = \arg \min_{\theta} g_N(\theta)' W_0  g_N(\theta)$.
\item Update $\hat{W} = \left(\frac{1}{N} \sum_{i=1}^N (g(w_i,\hat{\theta}) - g_N(\hat{\theta}))  \, (g(w_i,\hat{\theta}) - g_N(\hat{\theta}))'\right)^{-1}$
\item Solve $\hat{\theta}_{GMM} = \arg \min_{\theta}\, g_N(\theta)' \, \hat{W} \, g_N(\theta)$.
\item Compute $D(\hat{\theta}_{GMM})$ and $S(\hat{\theta}_{GMM})$ and compute standard errors.
\end{enumerate}
\end{frame}

\begin{frame}{Estimating the Variance Matrix}
For the linear IV estimator when $i$ is independent then $g(w_i,\theta) = z_i \epsilon_i$ and $E[z_i \epsilon_i]=0$
\begin{eqnarray*}
\hat{S}=\frac{1}{N} \sum_{i=1}^N z_i z_i' \epsilon_i^2
\end{eqnarray*}
\begin{itemize}
\item When there is homoskedastic variance $E[\epsilon_i^2 | z_i] = \sigma^2$ and the covariance of the moments becomes $\frac{\sigma^2}{N} \sum_{i=1}^N z_i z_i'$.
\item Because scaling weighting matrix by a constant has no effect on the maximum this is equivalent to the 2SLS weight matrix: $\sum_{i=1}^N z_i z_i'$ or $Z'Z$
\begin{itemize}
\item  2SLS is only the efficient estimator under \alert{homoskedasticity}.
\item  Likewise, if all regressors are exogenous then $X=Z$ and we are left with the GMM formula coincides with the covariance for heteroskedasticity robust standard errors.
\item Similarly, when appropriate we can consider extensions such as \alert{clustered standard errors} which are robust to weaker forms of independence.
\end{itemize}
\end{itemize}
\end{frame}

\begin{frame}{Estimating the Variance Matrix}
 As a practical matter, we should always use the \alert{sandwich} form when calculating the GMM standard errors, rather than the simpler \alert{bread} version which is only correct at $\theta_0$ under asymptotic optimality conditions.
\end{frame}


\begin{frame}{Example: Gravity Equation}
\begin{itemize}
\item An important set of models in international trade talk about \alert{Gravity Equations}
\item They are called gravity because trade declines with distance (or distance$^2$).
\end{itemize}
\begin{align*}
T_{ij}  = \alpha_0 Y_i^{\alpha_1} Y_j^{\alpha_2} D_{ij}^{\alpha_3} \eta_{ij}
\end{align*}
Take Logs
\begin{align*}
\ln T_{ij}  =  \ln \alpha_0  +\alpha_1 \ln Y_i + \alpha_2 \ln Y_j + \alpha_3 \ln D_{ij} + \ln \eta_{ij}
\end{align*}
\begin{itemize}
\item $T_{ij}$  (Exports from $i$ to $j$)
\item $(Y_i,Y_j)$ GDP of each country
\item $D_{ij}$ distance between two countries
\end{itemize}
\end{frame}


\begin{frame}{Example: Gravity Equation}
If the moment condition holds then everything is good:
\begin{align*}
E[ln (\eta_{ij}) | Y_i, Y_j D_{ij} ]=0
\end{align*}
Some problems
\begin{itemize}
\item Lots of Zeros in $T_{ij}$ (so we can't take logs).
\item If $( Y_i, Y_j D_{ij},\eta_{ij})$ has heteroskedasticity then moment condition is violated.
\item Why?  Expectation is \alert{linear operator} but $\log(\cdot)$ not so much.
\end{itemize}
\end{frame}

\begin{frame}{Gravity: Part 2}
Rearrange things so that:
\begin{align*}
T_{i j}&=\exp \left(\beta_{0}+\alpha_{1} \log \left(Y_{i}\right)+\alpha_{2} \log \left(Y_{j}\right)+\alpha_{3} \log \left(D_{i j}\right)\right) \eta_{i j} \\
T_{i j}&=\exp \left(x_{i} \beta\right) \eta_{i j}
\end{align*}
This gives us our moment condition:
\begin{align*}
E[ T_{i j} - \exp \left(x_{i} \beta\right) | x_i]=0
\end{align*}
This works as long as we are okay with \alert{proportional variance}

\end{frame}
\section*{Thanks!}

\end{document}